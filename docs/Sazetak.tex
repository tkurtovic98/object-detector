\begin{sazetak}
    Ovaj rad predstavlja neke arhitekture modela koji se koriste prilikom 
    detektiranja objekata. Razvoj ove domene u zadnjih par godina bio je 
    značajan, što potvrđuje složenost i preciznost dostupnih modela. RCNN-a bio je prvi značajan model u ovom području.
    Modeli koji su kasnije razvijeni su YOLO i SSD modeli i oni su pomaknuli granice i u brzini i u preciznosti detektiranja.
    Uključenost tvrtke Google u razvoj i distribuciju unaprijed treniranih modela također čini postupak treniranja jednostavnijim za korisnika, eliminirajući brigu o 
    konfiguraciji arhitekture. Nakon treniranja modela slijedi implementacijski postupak u kojem se 
    objašnjavaju svi važni koraci koji su omogućili pokretanje modela unutar mobilne platforme.
    
    \kljucnerijeci{Android, Tensorflow, Flutter, SSD MobileNet}
    \end{sazetak}
    
\engtitle{Object detection}
\begin{abstract}
    This paper presents some model architectures used in the field of object detection. 
    The development of this domain in the last couple of years has been significant, 
    which the complexity and precision of the available models confirm. RCNN was the first significant model in this area.
    After it other models such as the YOLO and SSD model were developed and have shifted boundaries both in speed and in detection accuracy.
    Also, Google's involvement in the development and distribution
    of pretrained models makes the training process easier for the user, eliminating the concern about
    architecture configuration. After training the model follows the implementation process in which
    all the important steps that enabled the launch of the model within the mobile platform are explained.

    \keywords{Android, Tensorflow, Flutter, SSD MobileNet}
\end{abstract}
    