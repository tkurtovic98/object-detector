\chapter{Uvod}
\section{Računalni vid}
Računalni vid grana je umjetne inteligencije u kojoj se računalo "uči" interpretirati slike. 
Danas se primjenjuje u:
\begin{itemize}
    \item Optičkom prepoznavanju znakova
    \item Sigurnosnim sustavima
    \item Medicini
\end{itemize}
Započinje početkom 1970-tih, međutim tada ljudi nisu znali niti što bi trebali učiniti kako bi razvili ovakve sustave niti koliko
bi takav razvoj trajao. Tadašnji vrsni znanstvenici i profesori smatrali su da će problem moći riješiti "unutar jednog ljeta, tako da priključe kameru na računalo". \citep{szeliski2010computer}

\section{Detekcija objekata}
Detekcija objekata podgrana je računalnog vida te označava postupak kojim računalo nastoji odrediti razred
i položaj objekta na slici. Razred ili klasa \engl{Class} označava vrstu kojoj objekt na slici pripada. Razredi mogu biti generalizirani ili specijalizirani. 
Na primjer, "pas" može biti jedan razred, ali isto tako može biti i "bulldog", "pekinezer", "bigl" itd.
Ovakav postupak zapravo objedinjuje postupak klasifikacije objekata i lokalizacije objekata. 
Klasifikacija je postupak određivanja razreda, a lokalizacija je postupak određivanja pozicije objekta.
Složenost ovog postupka nalazi se u činjenici da je za određivanje razreda kojem objekt pripada potrebna golema količina podataka 
iz stvarnoga svijeta. Detekcija objekata može se odvijati u stvarnome vremenu \engl {real-time}  ili naknadnom analizom na slici.


\section{Cilj}
Cilj ovoga rada je proučiti dostupne metode detekcije objekata na slikama koristeći duboko učenje i implementacija jedne metode unutar mobilnog okruženja. 
 
