\chapter{Zaključak}

U ovome radu glavna ideja bila je objasniti i implementirati model detekcije objekata. Detektiranje objekata
u stvarnome vremenu veoma je složen postupak koji zahtjeva mnogo računalne snage. Međutim, zadnjih godina
istraživači su došli do modela koji brzo i precizno daju rezultate. Početak razvoja modela za detektiranje objekata 
može biti pripisan prvom RCNN-u, jer je on bio temelj za daljnje napretke u ovome području. \newline
Premda se promatranjem i čitanjem tuđih radova može jako puno naučiti, za praktično znanje i razumijevanje
bilo je potrebno pronaći neki model te ga implementirati unutar mobilne platforme. Budući da je SSD jedan od modela
s najvećom kvalitetom, on je poslužio kako bi se na njemu izvelo treniranje. Korištenje unaprijed treniranog modela preporučena je 
metoda razvoja, jer znatno smanjuje vrijeme potrebno da bi se razvila aplikacija. Uz pomoć brojnih alata i sučelja te koristeći 
Googleovu platformu na oblaku, treniranje i validacija navedenog modela nisu bili toliko zahtjevni koliko je bila zahtjevna sama 
konfiguracija svih potrebnih datoteka. \newline \newline
Nakon implementacije navedenog modela uz pomoć Flutter radnog okvira, bilo je moguće testirati samu aplikaciju na stvarnim, još
neviđenim fotografijama. Razvijanje uz pomoć Flutter radnog okvira bio je dobar odabir, jer se uz manje preinake izgled aplikacije može prilagoditi 
korisnicima IOS mobilne platforme. Ovakva vrsta aplikacije može imati široku primjenu, jer je postupak treniranja i implementiranja jednak za 
bilo koju vrstu objekata koju želimo moći detektirati. Daljnji razvoj ove aplikacije može, na primjer, uključiti detektiranje večeg broja pasa i mačaka, kako bi se mogla stvoriti baza podataka svih pasa i mačaka te njihovih vlasnika. 
Time bi se, u slučaju da se kućni ljubimac izgubi, omogučilo lakše pronalaženje kako vlasnika tako i informacija o izgubljenoj životinji.